\documentclass[10pt]{article}

\usepackage{german}
\usepackage{amscd}
\usepackage{amssymb}
\usepackage[dvips]{graphicx}  % cf. p 63 und 73
\usepackage{graphics}
\usepackage{epsfig}
\usepackage{multicol}
\usepackage{rawfonts}

\pagestyle{empty}

\setlength{\topmargin}{0cm} \setlength{\textheight}{25cm}
\setlength{\textwidth}{15.9cm} \setlength{\oddsidemargin}{0cm}
\setlength{\evensidemargin}{0cm} \setlength {\parindent}{0pt}
\setlength{\parindent}{0mm}
\newtheorem{A}{Aufgabe}
\newtheorem{Lo}{L\"osung}

\newcommand{\N}{{\mathbb N}}
\newcommand{\C}{{\mathbb C}}
\newcommand{\Q}{{\mathbb Q}}
\newcommand{\R}{{\mathbb R}}
\newcommand{\Z}{{\mathbb Z}}
\newcommand{\vs}{\vspace{1mm}}
\newcommand{\Vs}{\vspace{4mm}}
\newcommand{\VS}{\vspace{1cm}}
\newcommand{\hs}{\hspace{1cm}}
\newcommand{\nin}{\noindent}

\input prepictex
\input pictex
\input postpictex

\begin{document}

%\newfont\fiverm{cmr5}
\setplotsymbol({\fiverm .})
\setlength{\unitlength}{1mm}

\begin{sffamily} % cf. p94 and 153
Klasse DP05a \hfill WiSe 06/07  ung\\

%\VS
\begin{center}
{\bf NMMa3 \hs Serie 4}
\end{center}
%\vs


\begin{A}  \end{A}
Bilden Sie das Dreieck $P_1 P_2 P_3$ auf das Dreieck $Q_1 Q_2 Q_3$ ab.\\

\beginpicture
\setlinear \setcoordinatesystem units <.9mm,.9mm> \setplotarea x
from -50 to 50, y from -20 to 30
%\put {\line(1,1){4}} at 0 1
%\put {\line(1,-2){2}} at 4 5
\put {\vector(1,0){30}}   [Bl] at -30 10 \put {\vector(0,1){30}}
[Bl] at -30 10 \put {\vector(1,0){30}}   [Bl] at 40 10 \put
{\vector(0,1){30}}   [Bl] at 40 10
%\put {\vector(-2,-1){5}} [Bl] at 40 40
%\put {\vector(-1,-1){35}} [Bl] at 40 40
%\put {\vector(-2,-5){20}} [Bl] at 40 40
%\put {\vector(-6,-1){60}} [Bl] at 40 40
\put {$\bullet$}  at  -30 10 \put {$\bullet$}  at  -15 35 \put
{$\bullet$}  at  -5  -10 \put {$P_1(0,\,0)$} [Bl] at -43 7 \put
{$P_2(x_2,\,y_2)$} [Bl] at -10 -14 \put {$P_3(x_3,\,y_3)$} [Bl] at -15
37 \plot -30 10 -15 35 / \plot -15 35 -5 -10 / \plot  -5 -10 -30 10
/ \put {$x$}  [Bl] at 0 7 \put {$y$}  [Bl] at -33 40 \put
{$\bullet$}  at  40 10 \put {$\bullet$}  at  40 25 \put {$\bullet$}
at  55 10
%\put {\line(0,1){20}} [Bl] at -20 -10
\plot  55   10 40   25 /
%\plot -20 -10  0   0 /
%\plot   0   0 0   40 /
%\plot -20 -10 20 -10 /
%\plot -20 -10 -20 30 /
%\plot -20  30 20 30 /
%\plot -20  30 0  40 /
%\plot   0  40 40 40 /
%\plot  20 -10 40 0 /
%\plot  40   0 40 40 /
%\plot 20 -10  20 30 /
%\plot 20 30   40 40 /
%\put {$y$} [Bl] at 50 42
%\put {$z$} [Bl] at 41 50
%\put {$x$} [Bl] at 36 35
%\put {$P$} [Bl] at 21 28
\put {$Q_1(0,\,0)$} [Bl] at 30 6 \put {$Q_2(1,\,0)$} [Bl] at 53 6 \put
{$Q_3(0,\,1)$} [Bl] at 35 27 \put {$\xi_1$}  [Bl] at 70 7 \put
{$\xi_2$}  [Bl] at 37 40
%\put {$O$} [Bl] at 41 41

%\plot 9 -8.5 10 -5 20 30  24 44 /
%\plot -1 -7.5 33 36.5 /
%\plot 2 -3 0 0 -20 30 -26 39 /
%\put {$h$} [Bl] at  19 35
%\put {$g$} [Bl] at -23 36

%\put {\circle*{1.2}} [Bl] at 20 30
%\put {$\bullet$} at 10 -5
%\put {\circle*{1.2}} [lb]  at -20 30

\endpicture

\begin{A}  \end{A}
Gegeben seien die folgenden Abbildungen des $\R^3$ in sich:
\begin{center}
\begin{tabular}{l l}
 $\mathcal{F}_1\::$ & Spiegelung an der Ebene $x_1 = x_2$\\
 $\mathcal{F}_2\::$ & Spiegelung an der Ebene $x_1 = 0$\\
 $\mathcal{F}_3\::$ & Drehung um die  $x_3 - $Achse um den Winkel $\varphi = \frac{\pi}{6}$\\
\end{tabular}
\end{center}
\begin{enumerate}
\item[a)] Bestimmen Sie die Abbildungsmatrizen $A_k$ bzgl. der
Standardbasis $\Sigma_e$ der Abbildungen $\mathcal{F}_k$, $k = 1, 2,
3$. \item[b)] Verifizieren Sie mit Hilfe der Matrizen $A_1$, $A_2$
und $A_3$ die Gleichung
\[
 \mathcal{F}_3\circ\mathcal{F}_3\circ\mathcal{F}_3 = \mathcal{F}_2\circ\mathcal{F}_1
\]
geometrische Interpretaion
\end{enumerate}

\begin{A}  \end{A}
Wir betrachten die {\em lineare} Abbildung
$\mathcal{F}:\:\R^3\longmapsto\R^2$, welche die drei Basisvektoren
$e_1$, $e_2$ und $e_3$ des $\R^3$ wie folgt abbildet:
\[
 \mathcal{F}(e_1) = \left( \begin{array}{r}
                            3\\-1
                           \end{array} \right)\quad
 \mathcal{F}(e_2) = \left( \begin{array}{r}
                            2\\1
                           \end{array} \right)\quad
 \mathcal{F}(e_3) = \left( \begin{array}{r}
                            -1\\3
                           \end{array} \right)
\]
\begin{enumerate}
\item[a)] Bestimmen Sie das Bild $P'$ des Punktes $P(6,\, 2,\, 3)$ unter
$\mathcal{F}$. \item[b)] Bestimmen Sie alle Punkte des $\R^3$, die
auf $Q'(2,\,1)$ abgebildet werden. \item[c)] Geben Sie den {\em Kern},
d.h. $\mathcal{F}(x) = 0$ von $\mathcal{F}$ an.
\end{enumerate}

\begin{A}  \end{A}
Betrachten Sie folgende {\em lineare} Abbildungen im $\R^3$ bzgl.
der Standardbasis $\Sigma_e$
\begin{enumerate}
\item[a)] Spiegelung $S$ an der $xz - $Ebene.
\item[b)] Drehung $D_x$ um die $x - $Achse, Drehwinkel $\varphi_x$\\
          Drehung $D_y$ um die $y - $Achse, Drehwinkel $\varphi_y$\\
      Drehung $D_z$ um die $z - $Achse, Drehwinkel $\varphi_z$
\item[c)] Zusammensetzung: zuerst Spiegelung $S$, dann $D_z$, dann
$D_x$ und schliesslich $D_y$, speziell f\"ur $\varphi_x =
\frac{\pi}{4}$ und $\varphi_y = \varphi_z =  \frac{\pi}{2}$
\end{enumerate}


\newpage


\begin{center}
{\bf NMMa3 \hs L\"osungen \hs Serie 4}
\end{center}

\VS
\begin{Lo}  \end{Lo}
Diese Abbildung (Grundaufgabe der Methode der finiten Elemente) hat
die Abbildungsmatrix
\[
 A = \frac{1}{d}\:\left( \begin{array}{r r}
                         y_3 & -x_3\\
             -y_2 & x_2
                         \end{array} \right)\qquad d := x_2\:y_3 - x_3\:y_2
\]

%\VS
\begin{Lo}  \end{Lo}
\begin{enumerate}
\item[a)]
\[
 A_1 =  \left(\begin{array}{r r r}
                                   0 & 1 & 0\\
                   1 & 0 & 0\\
                   0 & 0 & 1
                                \end{array}
\right)\quad
 A_2 =  \left(\begin{array}{r r r}
                                   -1 & 0 & 0\\
                   0 & 1 & 0\\
                   0 & 0 & 1
                                \end{array}
\right)\quad A_3 =  \left(\begin{array}{r r r}
                                   \frac{\sqrt{3}}{2} & -\frac{1}{2} & 0\\
                   \frac{1}{2} & \frac{\sqrt{3}}{2} & 0\\
                   0 & 0 & 1
                                \end{array}
\right)
\]
mit Hilfe der Bilder von $e_k$, $k = 1, 2, 3$ \item[b)]
\[
  A^3_3 = \left(\begin{array}{r r r}
                                   0 & -1 & 0\\
                   1 & 0 & 0\\
                   0 & 0 & 1
                                \end{array}
\right) = A_2\cdot A_1\quad\textrm{Reihenfolge!} = \textrm{Drehung
um die $x_3 - $Achse, Drehwinkel $\varphi = \frac{\pi}{2}$}
\]
\end{enumerate}

%\VS
\begin{Lo}  \end{Lo}
\[
 F = \left(\begin{array}{r r r}
                                   3 & 2 & -1\\
                   -1 & 1 & 3
                                \end{array}
\right)\quad\textrm{bzgl. $\Sigma_e$}
\]
\[
\textrm{a)}\: F(P) = P',\; P'(19,\,5)\quad \textrm{b)}\:g:\; r =
\left(\begin{array}{r}
                             x\\y\\z
                            \end{array}
                             \right) =
                           \left(\begin{array}{r}
                             0\\1\\0
                            \end{array}
                             \right) + \mu
                 \left(\begin{array}{r}
                             7\\-8\\5
                            \end{array}
                             \right)\;\mu\in\R\quad
\]
\[
\textrm{c)}\: F x = 0:\;Kern(F)= span\left\{ \left(\begin{array}{r} 7 \\\ -8 \\5 \end{array} \right)
\right\}
\]

\begin{Lo}  \end{Lo}
\[
\textrm{a)}\: S = \left(\begin{array}{r r r}
                                   1 & 0 & 0\\
                   0 & -1 & 0\\
                   0 & 0 & 1
                                \end{array}
\right)\; S^2 = I_3
\]
\[
\textrm{b)}\:D_x = \left(\begin{array}{r r r}
                                   1 & 0 & 0\\
                   0 & \cos{\varphi_x} & -\sin{\varphi_x}\\
                   0 & \sin{\varphi_x} & \cos{\varphi_x}
                                \end{array}
\right)\quad D_y = \left(\begin{array}{r r r}
                                   \cos{\varphi_y} & 0 & \sin{\varphi_y}\\
                   0 & 1 & 0\\
                   -\sin{\varphi_y} & 0 & \cos{\varphi_y}
                                \end{array}
\right)\quad D_z = \left(\begin{array}{r r r}
                                   \cos{\varphi_z} & -\sin{\varphi_z} & 0\\
                   \sin{\varphi_z} &  \cos{\varphi_z} & 0\\
                   0 & 0 & 1
                                \end{array}
\right)
\]
\[
\textrm{c)}\: A = \left(\begin{array}{r r r}
                                   \frac{\sqrt{2}}{2} & 0 &  \frac{\sqrt{2}}{2}\\
                   \frac{\sqrt{2}}{2} & 0 & -\frac{\sqrt{2}}{2}\\
                   0 & -1 & 0
                                \end{array}
\right) = D_y\cdot D_x\cdot D_z\cdot S
\]




%
%\newpage
%
%%33333333333333333333333333333333333333333333333333
% weitere Aufgaben
% ---------------------------------------------------


%\begin{A}  \end{A}
%Sei $E$ eine Ebene im $\R^3$ durch den Nullpunkt orthogonal zum
%Vektor $n = \left( 1,\;-1,\;2 \right)^T$
%\begin{enumerate}
%\item[a)] Geben Sie die Abbildung $\mathcal{P}$ an, die jeden
%Vektor $x\in\R^3$ orthogonal auf die Ebene $E$ projiziert.
%\item[b)] Stellen Sie die Abbildung $\mathcal{P}$ durch eine
%Abbildungsmatrix $P$ bez\"uglich der Standardbasis $\Sigma_e:\;( =
%e_1, e_2, e_3)$ dar. \item[c)] Was f\"ur Eigenschaften hat $P\:$?
%\item[d)]  W\"ahlen Sie ein neues Koordinatensystem $\Sigma'$ so,
%dass die neue Abbildungsmatrix $P'$ m\"oglichst einfach wird.
%\end{enumerate}

%\begin{A}  \end{A}
%\begin{enumerate}
%\item[a)] Welche der folgenden Vektor-Tripel bilden eine Basis im
%$\R^3$:
%\begin{center}
%\begin{tabular}{r r r}
%$f_1 = \left( 1,\;0,\;1 \right)^T$ & $f_2 = \left( 0,\;1,\;1 \right)^T$ & $f_3 = \left( 1,\;1,\;0 \right)^T$\\
%$g_1 = \left( 1,\;2,\;1 \right)^T$ & $g_2 = \left( 2,\;1,\;1 \right)^T$ & $g_3 = \left( 4,\;5,\;3 \right)^T$\\
%$h_1 = \left( 3,\;2,\;-4 \right)^T$ & $h_2 = \left( 2,\;4,\;6
%\right)^T$ & $h_3 = \left( 17,\;6,\;4 \right)^T$
%\end{tabular}
%\end{center}
%\item[b)] Der Vektor $v\in\R^3$ hat bzgl. der Standardbasis
%$\Sigma_e$ die Koordinaten $\left( 1,\;2,\;3 \right)^T$. Bestimmen
%Sie in den F\"allen, in denen obige Vektoren eine Basis bilden, die
%Koordinaten von $v$.
%\end{enumerate}

%\begin{A}  \end{A}
%\beginpicture
%\setlinear \setcoordinatesystem units <.9mm,.9mm> \setplotarea x
%from -50 to 50, y from -20 to 30
%%\put {\line(1,1){4}} at 0 1
%%\put {\line(1,-2){2}} at 4 5
%\put {\vector(1,0){10}}   [Bl] at 40 40 \put {\vector(0,1){10}} [Bl]
%at 40 40 \put {\vector(-2,-1){5}} [Bl] at 40 40
%%\put {\vector(-1,-1){35}} [Bl] at 40 40
%%\put {\vector(-2,-5){20}} [Bl] at 40 40
%%\put {\vector(-6,-1){60}} [Bl] at 40 40
%\put {$\bullet$}  at  40 40
%%\put {\line(0,1){20}} [Bl] at -20 -10
%\plot   0   0 40   0 / \plot -20 -10  0   0 / \plot   0   0 0   40 /
%\plot -20 -10 20 -10 / \plot -20 -10 -20 30 / \plot -20  30 20 30 /
%\plot -20  30 0  40 / \plot   0  40 40 40 / \plot  20 -10 40 0 /
%\plot  40   0 40 40 / \plot 20 -10  20 30 / \plot 20 30   40 40 /
%\put {$y$} [Bl] at 50 42 \put {$z$} [Bl] at 41 50 \put {$x$} [Bl] at
%36 35
%%\put {$P$} [Bl] at 21 28
%\put {$\vec a$} [Bl] at 23 -5 \put {$\vec b$} [Bl] at 3 5 \put
%{$\vec c$} [Bl] at 3 35 \put {$O$} [Bl] at 41 41
%
%%\plot 9 -8.5 10 -5 20 30  24 44 /
%%\plot -1 -7.5 33 36.5 /
%%\plot 2 -3 0 0 -20 30 -26 39 /
%%\put {$h$} [Bl] at  19 35
%%\put {$g$} [Bl] at -23 36
%
%%\put {\circle*{1.2}} [Bl] at 20 30
%%\put {$\bullet$} at 10 -5
%%\put {\circle*{1.2}} [lb]  at -20 30
%
%\endpicture
%
%Gegeben: $\vec a$, $\vec b$ sowie $\vec c$ im Einheitsw\"urfel,
%Kantenl\"ange $ = 1$, cf. Figur und ein Vektor $\vec d =
%\left(1,\;1,\;1  \right)^T$
%\begin{enumerate}
%\item[a)] $\vec d$ als Linearkombination von $\vec a$, $\vec b$
%und $\vec c$. \item[b)] Bestimmen Sie eine o.n. Basis $\Sigma'$ mit
%$b_1 = $Vielfaches von $\vec a$, $b_1$ und $b_2$ liegen in der von
%$\vec a$ und $\vec b$ aufgespannten Ebene. \item[c)] $\vec d$ in der
%neuen Basis $\Sigma'$
%\end{enumerate}

%\begin{A}  \end{A}
%Gegeben sind die beiden Vektoren $a = \left( 2,\;1 \right)^T$ und $b = \left( -3,\;6 \right)^T$. Diese beiden Vektoren sind orthogonal.\\
%Gesucht ist $v =  \left( -20,\;16 \right)^T$ als Linearkombination
%von $a$ und $b$.


%3333333333333333333333333333333333333333333333333333333333
%\begin{A}  \end{A}
%Gegeben seien die Vektoren
%
%$$
%   v_1 = \left(  \begin{array}{r} 1\\1\\1\\1
%         \end{array} \right), \quad
%   v_2 = \left(  \begin{array}{r} 0\\2\\-3\\0
%         \end{array} \right), \quad
%   v_3 = \left(  \begin{array}{r} 2\\-4\\6\\0
%         \end{array} \right), \quad
%   v_4 = \left(  \begin{array}{r} 1\\0\\0\\1
%         \end{array} \right), \quad
%   v_5 = \left(  \begin{array}{r} 2\\0\\0\\0
%         \end{array} \right), \quad
%   v_6 = \left(  \begin{array}{r} 0\\0\\0\\0
%         \end{array} \right)
% $$
%Bestimmen Sie alle Teilmengen $\mathbb{T}$ der Menge $\mathbb{S} =
%\left\{ v_1, v_2, v_3, v_4, v_5, v_6 \right\}\subset\R^4$, die eine
%Basis des $\R^4$ bilden.

%\begin{A}  \end{A}
%Im $\R^4$ wird durch die Vektoren $\left\{ a_1, a_2, a_3\right\}$
%ein Unterraum $U$ und durch die Vektoren $\left\{b_1, b_2,
%b_3\right\}$ ein Unterraum $V$ aufgespannt, also $U = span \left\{
%a_1, a_2, a_3\right\}$ und $V = span \left\{b_1, b_2, b_3\right\}$.
%
%\[
%   a_1 = \left(  \begin{array}{r} 2\\-1\\3\\5
%         \end{array} \right), \quad
%   a_2 = \left(  \begin{array}{r} 5\\-2\\5\\8
%         \end{array} \right), \quad
%   a_3 = \left(  \begin{array}{r} -5\\3\\-8\\-13
%         \end{array} \right), \quad
%   b_1 = \left(  \begin{array}{r} 4\\1\\-2\\-4
%         \end{array} \right), \quad
%   b_2 = \left(  \begin{array}{r} 7\\2\\-6\\-9
%         \end{array} \right), \quad
%   b_3 = \left(  \begin{array}{r} 3\\0\\0\\-1
%         \end{array} \right)
%\]
%
%Bestimmen Sie je eine Basis der Unterr\"ume $U + V$ sowie $U\cap V$.

%\begin{A}  \end{A}
%\begin{enumerate}
%\item[a)] Zeigen Sie, dass durch folgende Abbildung ein
%Skalarprodukt auf dem Vektorraum $V = \R^{n\times n}$ der $n\times n
%-$Matrizen definiert ist:
%$$
% (A, B) := Spur(A B^T),\quad A, B\in\R^{n\times n}
%$$
%\item[b)] Sei $D$ eine $n\times n -$Diagonalmatrix mit positiven
%Diagonalelementen. Behauptet wird, dass folgende Vorschrift
%\[
% (x, y):= x^T D y,\quad x, y\in\R^n
%\]
%ein Skalarprodukt definiert. Beweisen Sie diese Behauptung.
%\end{enumerate}

%\begin{A}  \end{A}
%Auf dem Vektorraum $\mathcal{P}_2$ der Polynome vom Grad $\le 2$ ist
%das Skalarprodukt
%\[
% (p, q) := \int_0^1{p(x) q(x)\:dx}
%\]
%definiert.
%\begin{enumerate}
%\item[a)] Orthonormieren Sie die Standardbasis $\left\{1, x, x^2
%\right\}$ von $\mathcal{P}_2$ bez\"uglich diesem Skalarprodukt.
%Gram-Schmidt \item[b)] Projizieren Sie die Funktion $f(x) = e^x$
%orthogonal auf $\mathcal{P}_2$.
%\end{enumerate}


%3333333333333333333333333333333333333333333333333333333333333333333


\end{sffamily}
\end{document}
