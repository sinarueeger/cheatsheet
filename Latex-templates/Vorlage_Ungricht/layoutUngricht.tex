\documentclass[a4paper,10pt, leqno]{scrartcl}                  % Siehe f{\"u}r spezielle Anwendungen scrguide.pdf

% ftp.dante.de/tex-archive/macros/latex/contrib/koma-script/scrguide.pdf

\usepackage[ngerman]{babel}
%\usepackage{textcomp}
\usepackage[ansinew]{inputenc}

%\usepackage[dvips]{graphicx}  % cf. p 63 und 73
%\usepackage{graphics}
%\usepackage{epsfig}
%\usepackage{float}

%\usepackage{epsf}
%\usepackage{epsfig}
%\usepackage{graphicx}

%\usepackage{graphicx}

%\usepackage{fancyhdr}
%\usepackage{amscd}
\usepackage{amssymb}
\usepackage{amsmath}
\usepackage[a4paper, left=3cm, right=3cm, top=3.5cm,    % Hier lassen sich die Randabst{\"a}nde einstellen
bottom=2.5cm]{geometry}
\usepackage[german]{varioref}                           % F{\"u}r Verweise mit deutschem Text
\usepackage{rawfonts}


%\usepackage[all]{xy}           % f�r Diagramme, cf. p 80 ff

% Pagestyle festlegen
% ===================
%\renewcommand{\footrulewidth}{0.4pt}                    % F{\"u}r eine Linie am unteren Seitenrand
%\pagestyle{fancy}%
%\lhead{\footnotesize{MLAN1}}            % Bezeichnung des Moduls
%\chead{\textit{\footnotesize{\leftmark}}}
%\rhead{\footnotesize{\today}}%
%\rhead{\footnotesize{\thepage}}%
%\rfoot{\footnotesize \thepage}%
%\rfoot{\small ung/ \texttt{\jobname.tex}}    % fname automatisch!
%\cfoot{}%
%\lfoot{\footnotesize{Y.Blattmann DP1a}}%
%\lfoot{\footnotesize{ZHW}}

\setlength{\parindent}{0cm}                             % Es wird kein Einzug im Neuen Absatz gemacht
%\setlength{\parskip}{1.5ex plus 0.5ex minus 0.5ex}     % Ein Variabler Abstand zwischen den Abs{\"a}tzen

% Grafik einbinden Syntax: (Prozent der Breite, Name des File, Beschriftung, Label)
% m{\"o}gliche Formate sind .jpeg .tiff .png
% =================================================================================
\newcommand{\fig}[4]{%
    \begin{figure}[hbt!]                                % h = here, t = top, b = bottom, p = page
        \centering
        \includegraphics[width=#1\textwidth]{#2}
        \caption{\textit{\footnotesize{#3}}}
        \label{#4}
    \end{figure}
}

% Neue Definitionen:
% ==================
\newcommand{\dif}{\,\mbox{d}}                           % F{\"u}r ein gerades d bei integral von dx
\newcommand{\difq}[2]{\frac{\dif #1}{\dif #2}}          % Ableitungskoeffizient z.B dy/dt
\newtheorem{satz}{Satz}[section]                        % Aufz{\"a}hlungen, orientieren sich an der Sectionnummer
\newtheorem{kor}{Korollar}[section]
\newtheorem{Def}{Definition}[section]
\newtheorem{Fig}{Figur}[section]
\newtheorem{folg}{Folgerungen}[section]
\newtheorem{aufg}{Aufgabe}[section]
\newtheorem{loes}{L\"osung}[section]
\newtheorem{bsp}{Beispiel}[section]
\newtheorem{vor}{Vorgabe}[section]
\newtheorem{bem}{Bemerkung}[section]
\newtheorem{fra}{Frage}[section]
\newtheorem{ges}{Gesucht}[section]
\newcommand{\mL}{{\mathbf L}}
\newcommand{\N}{{\mathbb N}}                            % Zahlenmengen
\newcommand{\C}{{\mathbb C}}
\newcommand{\Q}{{\mathbb Q}}
\newcommand{\R}{{\mathbb R}}
\newcommand{\mS}{{\mathbb S}}
\newcommand{\V}{{\mathbb V}}
\newcommand{\Z}{{\mathbb Z}}

% dieses File kann mit \include{layout} eingebunden werden und enth{\"a}lt den ganzen Header!

% written by Yves Blattmann 02.06.04
