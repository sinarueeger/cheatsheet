%% Dokumentenklasse (Koma Script) -----------------------------------------
\documentclass[%
   %draft,     % Entwurfsstadium
   final,      % fertiges Dokument
	 % --- Paper Settings ---
   paper=a4,% [Todo: add alternatives]
   paper=portrait, % landscape
   pagesize=auto, % driver
   % --- Base Font Size ---
   fontsize=11pt,%
	 % --- Koma Script Version ---
   version=last, %
 ]{scrbook} % Classes: scrartcl, scrreprt, scrbook


% Encoding der Dateien (sonst funktionieren Umlaute nicht)
% Fuer Linux -> utf8
% Fuer Windows, alte Linux Distributionen -> latin1

% Empfohlen latin1, da einige Pakete mit utf8 Zeichen nicht
% funktionieren, z.B: listings, soul.
\usepackage[latin1]{inputenc}
%\usepackage[ansinew]{inputenc}
%\usepackage[utf8]{inputenc}
%\usepackage{ucs}
%\usepackage[utf8x]{inputenc}

%%% Preambel
\input{preambel/settings}
\input{preambel/preambel}
%
%%%% Neue Befehle
\input{macros/newcommands}
\input{macros/TableCommands}

%%% Silbentrennung
\input{preambel/Hyphenation}

%% Dokument Beginn %%%%%%%%%%%%%%%%%%%%%%%%%%%%%%%%%%%%%%%%%%%%%%%%%%%%%%%%

% - Deckblatt,
% - Inhaltsverzeichnis,
% - Hauptteil gegliedert z.B. in
%   Einleitung, Grundlagen, Experimente, Ergebnisse, Zusammenfassung
% - Literaturverzeichnis,
% - Abbildungsverzeichnis (ggf.),
% - Tabellenverzeichnis (ggf.),
% - Abk�rzungsverzeichnis (ggf.),
% - Formelverzeichnis (ggf.),
% - Anhang, (nicht mehr Bestandteil der Arbeit! Wird daher nicht bewertet)
% - Erkl�rung der Urheberschaft,

\begin{document}
% Deckblatt
\input{content/Titel}
\frontmatter

\cleardoublepage
% Inhaltsverzeichnis in den PDF-Links eintragen
% \pdfbookmark[1]{Inhaltsverzeichnis}{toc}
\tableofcontents

% Hauptteil
\mainmatter
% Demonstration der Vorlage
\input{content/Vorlage/demo.tex}

%\input{content/0-Einleitung}
%\input{content/1-Grundlagen}
%\input{content/2-Experimente}
%\input{content/3-Ergebnisse}
%\input{content/4-Zusammenfassung}


% Anhang (Bibliographie darf im deutschen nicht in den Anhang!)
\bibliography{bib/BibtexDatabase}
\clearpage
% Abbildungs- und Tabellenverzeichnis
\listoffigures
\listoftables
% Anhang
\appendix
% 'Anhang' ins Inhaltsverzeichnis
%\phantomsection
%\addcontentsline{toc}{part}{Anhang}

\input{content/Z-Anhang}

\IfDefined{printindex}{\printindex}
\IfDefined{printnomenclature}{\printnomenclature}



%% Dokument ENDE %%%%%%%%%%%%%%%%%%%%%%%%%%%%%%%%%%%%%%%%%%%%%%%%%%%%%%%%%%
\end{document}

