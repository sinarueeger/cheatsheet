%% KOMA-Script for articles
%\documentclass[paper=a4,fontsize=11pt,parskip=half,titlepage,headsepline,tocleft,english]{scrreprt} 
\documentclass[paper=a4,fontsize=11pt,parskip=half,titlepage,headsepline,tocleft,english]{scrreprt} 
%% bequemes Definieren des Layouts vgl. S. 29ff
\usepackage[top=25mm, bottom=25mm, left=25mm, right=25mm]{geometry}

%\usepackage{ngerman} %% Silbentrennung nach neuer deutschen Rechtschreibung

%% setzt alle Zeichen der CH-Tastatur richtig um inkl. �,� etc.
\usepackage[T1]{fontenc}       
\usepackage{lmodern} 
\usepackage[latin1]{inputenc}

%% Mathsymbole
\usepackage{amsmath}
\usepackage{amsfonts}


\usepackage[round,longnamesfirst]{natbib}

%% -------------------------------------------------------------------
%\usepackage[noaddress]{IDPscrartcl} 
\usepackage{IDPscrartcl}
%% -------------------------------------------------------------------

%%%%%%%%%%%%%%%%%%%%%%%%%%%%%%%%%%%%%%%%%%%%%%%%%%%%%%%%%%%%%%%%%%%%%%%%%%%%%%%%%%%%%%%%%
\newcommand{\MU}{\mathbf{\mu}}
\newcommand{\GAMMA}{\mathbf{\gamma}}
\newcommand{\X}{\mathbf{X}}
\newcommand{\x}{\mathbf{x}}
\newcommand{\Y}{\mathbf{Y}}
\newcommand{\Z}{\mathbf{Z}}
\newcommand{\EXP}[1]{\hbox{E}(#1)}
\newcommand{\VAR}[1]{\hbox{var}(#1)}
\newcommand{\COV}[1]{\hbox{cov}(#1)}
\newcommand{\QX}{\hbox{Q}(\x)}
\newcommand{\GammaSigGamma}{\GAMMA' \Sigma^{-1} \GAMMA}
\newcommand{\etaIK}{\eta_i^{[k]}}
\newcommand{\sumN}{\sum_{i=1}^n}
\newcommand{\deltaIK}{\delta_i^{[k]}}
\newcommand{\xiIK}{\xi_i^{[k]}}
\newcommand{\OneDivN}{\frac{1}{n}}
\newcommand{\dDivTwo}{\frac{d}{2}}
\newcommand{\NuDivTwo}{\frac{\nu}{2}}
\newcommand{\NuPlusDdivTwo}{\frac{\nu+d}{2}}
\newcommand{\LambdaDivTwo}{\frac{\lambda}{2}}
\newcommand{\ChiDivTwo}{\frac{\chi}{2}}
\newcommand{\PsiDivTwo}{\frac{\psi}{2}}
\newcommand{\OneDivTwo}{\frac{1}{2}}
\newcommand{\Bessel}[1]{\hbox{K}_{#1}}
\newcommand{\BesselLambda}{\hbox{K}_{\lambda}}
\newcommand{\BesselLambdaPlusOne}{\hbox{K}_{\lambda+1}}
\newcommand{\InvSigma}{\Sigma^{-1}}
\newcommand{\SqrtChiPsi}{\sqrt{\chi \psi}}
\newcommand{\MuSigGamma}{\MU,\Sigma,\GAMMA}
\newcommand{\LambdaChiPsi}{\lambda,\chi,\psi,\MuSigGamma}
\newcommand{\LambdaAbar}{\lambda,\abar,\MU,\Sigma,\GAMMA}
\newcommand{\GH}{\hbox{GH}_d(\LambdaChiPsi)}
\newcommand{\GHk}{\hbox{GH}_d(\lambda,\chi/k,k \psi,\MU,k \Sigma,k \GAMMA)}
\newcommand{\GHtransf}{\hbox{GH}_k(\lambda,\chi,\psi,
                      B \MU + \mathbf{b},B \Sigma B' ,B\GAMMA)}
\newcommand{\ThetaK}{\Theta^{[k]}}
\newcommand{\InvGamma}{\hbox{I}\Gamma}
\newcommand{\ghyp}{{\tt ghyp $\,$}}
\newcommand{\set}[1]{\left\{#1\right\}}
\newcommand{\sek}[1]{\left(#1\right)}

\newcommand{\ytilde}{(Q(\xvec)+\chi)}
\newcommand{\ltilde}{\lambda-\frac{d}{2}}

\newcommand{\xtildeIG}{(\mathbf{\gamma}'\Sigma^{-1}\mathbf{\gamma})}
\newcommand{\ytildeIG}{(Q(\xvec)+2\beta)}
\newcommand{\ltildeIG}{-\alpha-\frac{d}{2}}

\newcommand{\xtildeG}{(\mathbf{\gamma}'\Sigma^{-1}\mathbf{\gamma}+2\beta)}
\newcommand{\ytildeG}{(Q(\xvec))}
\newcommand{\ltildeG}{\alpha-\frac{d}{2}}

\newcommand{\betavec}{\boldsymbol{\beta}}
\newcommand{\muvec}{\boldsymbol{\mu}}
\newcommand{\xivec}{\boldsymbol{\xi}}
\newcommand{\gamvec}{\boldsymbol{\gamma}}
\newcommand{\QT}{\widetilde{Q}}
\newcommand{\SigT}{\boldsymbol{\widetilde{\Sigma}}}
\newcommand{\sigT}{\widetilde{\sigma}}
\newcommand{\dd}{\hbox{d}}
\newcommand{\abar}{\overline{\alpha}}

\newcommand{\PsiTransfGIG}{\GammaSigGamma + \psi}
\newcommand{\ChiTransfGIG}{\QX + \chi}
\newcommand{\LambdaTransfGIG}{\lambda - \dDivTwo}

\newcommand{\ChiTransfG}{\QX + \chi}

\newcommand{\PsiTransfIG}{\GammaSigGamma}

\newcommand{\norm}[1]{\left\Vert#1\right\Vert}
\newcommand{\dete}[1]{|#1|}

\newcommand{\customspace}{\\[3ex]}

\newcommand{\R}{\texttt{R}$\,$}

\renewcommand{\labelenumi}{(\roman{enumi})}
\numberwithin{equation}{section}
\numberwithin{table}{section}
\numberwithin{figure}{section}
\usepackage{babel}
\makeatother



\usepackage{C:/Programme/R/share/texmf/Sweave}}
\begin{document}
\title{St





















 Auswertung der Jalomed-Daten}
%\client{Preliminary draft}
\author{Sina R�eger}
\email{sina.rueeger@zhaw.ch}
\date{\today}

\maketitle

\begin{abstract}
sdfsdfsdf
\end{abstract}
\thispagestyle{empty}
\clearpage
\pagenumbering{arabic}
\tableofcontents{}
\clearpage
%-------------------------------------------------------------------------
%-------------------------- Introduction --------------------------------------
\section{Introduction}
sdfsdfsdf
%-------------------------------------------------------------------------
%-------------------------- Definition --------------------------------------
\section{Definition}
Facts about generalized hyperbolic (GH) distributions are cited according to
\cite{mcneil} chapter $3.2$.\\[1ex]


%\begin{equation} \label{eq:etadeltaxi}
%  \etaIK := \EXP{w_i \, | \, \x_i;\ThetaK},\,\,\,
%  \deltaIK := \EXP{w_i^{-1} \, | \, \x_i;\ThetaK},\,\,\,
%  \xiIK := \EXP{\ln w_i \, | \, \x_i;\ThetaK}.
%\end{equation}
We have to find the conditional density of $w_i$ given $\x_i$ to
be able to calculate these quantities (see (\ref{eq:conddistGIG})).
%\renewcommand{\labelenumi}{(\arabic{enumi})}
%\begin{enumerate}
%  \item{Select reasonable starting values for $\ThetaK$.
 
%%-------------------------------------------------------------------------
%\begin{appendix}
%%-------------------------- Shape ------------------------------
%%-------------------------------------------------------------------------
%\section{Shape of the univariate generalized hyperbolic distribution}
%\begin{figure}[!h]
%\begin{center}
%\setkeys{Gin}{width=0.75\textwidth}
%\begin{Schunk}
%\begin{Soutput}
%  ghyp, 2007, Institute of Data Analysis and Process Design, GPL
%\end{Soutput}
%\end{Schunk}
%\includegraphics{Generalized_Hyperbolic_Distribution-002}
%\caption{The shape of the univariate generalized hyperbolic density drawn
%         with different shape parameters $(\lambda,\abar)$. The location and
%         scale parameter $\mu$ and $\sigma$ are set to $0$ and $1$, respectively. The 
%         skewness parameter $\gamma$ is $0$ in the left column and $-1$ in the right
%         column of the graphics array.}
%\end{center}
%\end{figure}
%\clearpage
%-------------------------- Bessel ------------------------------
%-------------------------------------------------------------------------
%\begin{Soutput}
%Object of class 'ghypuv' (Univariate Generalized Hyperbolic)
%
%Model:
%Asymmetric Generalized Hyperbolic 
%
%Parameters:
%    lambda  alpha.bar          mu      sigma      gamma  
%-2.0000000  0.5000000  10.0000000  5.0000000  1.0000000  
%
%
%Slot 'data' is NULL.
%\end{Soutput}
%\begin{Sinput}
%> ghyp(lambda = -2, chi = 5, psi = 0.1, mu = 10:11, sigma = diag(5:6), 
%+     gamma = -1:0)
%\end{Sinput}
%\begin{Soutput}
%Object of class 'ghypmv' (Multivariate Generalized Hyperbolic)
%
%Model:
%Asymmetric Generalized Hyperbolic 
%
%Mixing parameter:
%   lambda       chi       psi  
%-2.000000  5.000000  0.100000  
%
%mu:
%[1] 10 11
%
%sigma:
%     [,1] [,2]
%[1,]    5    0
%[2,]    0    6
%
%gamma:
%[1] -1  0
%
%
%Slot 'data' is NULL.
%\end{Soutput}
%\end{Schunk}
%-------------------------------------------------------------------------
%-------------------------------------------------------------------------
%-------------------------------------------------------------------------
%\end{appendix}
%-------------------------------------------------------------------------
%\begin{thebibliography}{1}\label{Biblio}
%\bibitem{mcneil} \emph{Quantitative Risk Management: Concepts, Techniques and Tools} 
%                  by Alexander J. McNeil, R\"udiger Frey and Paul Embrechts \\
%                  Princeton Press, 2005
%\bibitem{QRMlib} \emph{S-Plus and \R Library for Quantitative Risk Management QRMlib} 
%                 by Alexander J. McNeil (2005) and Scott Ulman (\R-port) (2007)\\
%                 \url{http://www.math.ethz.ch/~mcneil/book/QRMlib.html}
%\bibitem{brw}    \emph{One-dimensional hyperbolic distributions}, Wolfgang Breymann, 
%                 unpublished, 2006 
%\end{thebibliography}
\end{document}

%%% Local Variables:
%%% mode: latex
%%% TeX-PDF-mode: t
%%% End: 

